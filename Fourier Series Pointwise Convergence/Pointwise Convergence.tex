\documentclass[12pt]{article}

\usepackage{answers}
\usepackage{setspace}
\usepackage{graphicx}
\usepackage{enumitem}
\usepackage{multicol}
\usepackage{mathrsfs}
\usepackage[margin=1in]{geometry} 
\usepackage{amsmath,amsthm,amssymb}

\renewcommand{\rmdefault}{ptm}

\newcommand{\N}{\mathbb{N}}
\newcommand{\Z}{\mathbb{Z}}
\newcommand{\C}{\mathbb{C}}
\newcommand{\R}{\mathbb{R}}

\DeclareMathOperator{\sech}{sech}
\DeclareMathOperator{\csch}{csch}
 
\newenvironment{theorem}[1][Theorem.]{\begin{trivlist}
\item[\hskip \labelsep {\bfseries #1}]}{\end{trivlist}}
\newenvironment{definition}[1][Definition.]{\begin{trivlist}
\item[\hskip \labelsep {\bfseries #1}]}{\end{trivlist}}
\newenvironment{proposition}[1][Proposition.]{\begin{trivlist}
\item[\hskip \labelsep {\bfseries #1}]}{\end{trivlist}}
\newenvironment{lemma}[1][Lemma.]{\begin{trivlist}
\item[\hskip \labelsep {\bfseries #1}]}{\end{trivlist}}
\newenvironment{exercise}[2][Exercise.]{\begin{trivlist}
\item[\hskip \labelsep {\bfseries #1}\hskip \labelsep {\bfseries #2.}]}{\end{trivlist}}

\newenvironment{solution}[2][Solution.]{\begin{trivlist}
\item[\hskip \labelsep {\bfseries #1}\hskip \labelsep {\bfseries #2.}]}{\end{trivlist}}
\newenvironment{problem}[2][Problem.]{\begin{trivlist}
\item[\hskip \labelsep {\bfseries #1}\hskip \labelsep {\bfseries #2.}]}{\end{trivlist}}
\newenvironment{question}[1][Question.]{\begin{trivlist}
\item[\hskip \labelsep {\bfseries #1}]}{\end{trivlist}}
\newenvironment{corollary}[1][Corollary.]{\begin{trivlist}
\item[\hskip \labelsep {\bfseries #1}]}{\end{trivlist}}
\newenvironment{hw}[2][HW.]{\begin{trivlist}
\item[\hskip \labelsep {\bfseries #1}\hskip \labelsep {\bfseries #2.}]}{\end{trivlist}}

%this next part is my footnote symbol
\usepackage[symbol*]{footmisc}
\DefineFNsymbolsTM{myfnsymbols}{% def. from footmisc.sty "bringhurst" symbols
  \textasteriskcentered *
  \textdagger    \dagger
  \textdaggerdbl \ddagger
  \textsection   \mathsection
  \textbardbl    \|%
  \textparagraph \mathparagraph
}%
\setfnsymbol{myfnsymbols}

\usepackage{mathptmx}

\begin{document}
 
% --------------------------------------------------------------
%                         Start here
% --------------------------------------------------------------
\title{Point-wise Convergence}
\author{Alvis Zhao}
\makeatletter
\hfil\parbox[t]{\textwidth}{\Large\bfseries\@title\\[0.5ex]\normalsize\bfseries\@author\\[0.5ex]\@date}\par
\makeatother

\section*{Point-wise Convergence}
Fourier series is defined to be \[ \sum\limits\hat{f}(\theta)e^{in\theta}
\]
with partial sums
\[S_Nf = \sum\limits_{-N}^N\hat{f}(\theta)e^{in\theta}
\]
Where the Fourier coefficient for $f \in L^1(\mathbb{T})$, $\mathbb{T} = [-\pi,\pi ]$ is defined by \[ \hat{f} = \frac{1}{2\pi}\int_\mathbb{T}f(\theta)e^{-in\theta}d\theta\]
To have a nicer form of the $S_Nf$ to work with, we want write it as the convolution of $f$ with Dirichlet kernel $D_N(\theta)= \sum_{-N}^Ne^{in\theta}$ by next proposition.

The convolution of two $L^1$ function is defined by 
\[
(f*g)(\theta)=\frac{1}{2\pi}\int_{\mathbb{T}}f(\phi)g(\theta-\phi)d\phi
\]
Following discussions will use Lebesgue's Dominated Convergence Theorem which is stated here without providing proof.
\begin{theorem}
\textit{Dominated convergence theorem}.
Let ${f_n}$ be a sequence in $L^1$ such that $f_n \rightarrow f$ a.e., and there exists a non-negative $g\in L^1$ such that $|f_n| \leqslant g$ for all n. Then $f\in L^1$ and $\int f = \lim_{n\rightarrow \infty}\int f_n$.
\end{theorem}

\begin{proposition}
Suppose that $\sum_{n\in\mathbb{Z}}|A_n| < \infty$ and we have an absolutely convergent Fourier series $K(\theta) = \sum_{n\in\mathbb{Z}}A_ne^{in\theta}$. Then for any $f \in L^1(\mathbb{T})$ we have \[(f*K)(\theta)=\sum\limits_{n\in\mathbb{Z}}A_n\hat{f}(n)e^{in\theta}\]
\end{proposition}
\begin{proof}
\[
f(\theta-\phi)K(\phi) = \sum A_ne^{in\phi}f(\theta-\phi)
\]
To apply  dominated convergence theorem, we need to check the RHS converges a.e. and bounded by a integrable function.
Since $f \in L^1$, $|f| \in L^1$, $K$ is bounded by $|\sum A_n|$, this series is bounded by $|f| |\sum A_n|$ and converges by assumption.
Thus, 
\begin{align*}
\int_{\mathbb{T}} \sum A_ne^{in\phi}f(\theta-\phi) &= \sum A_n\int_{\mathbb{T}} e^{in\phi}f(\theta-\phi)d\phi \\
&= \sum A_n\int_{\mathbb{T}} e^{in\phi}f(\theta-\phi)d\phi \\
&= \sum A_n\int_{\mathbb{T}} e^{in(\theta-\psi)}f(\psi)d\psi \\
&= \sum A_n\int_{\mathbb{T}} e^{in\theta}e^{-in\psi}f(\psi)d\psi \\
&= 2\pi \sum A_n \hat{f}(n) e^{in\theta}
\end{align*}
\end{proof}
Let $1_{[-N,N]}(n)$ be the indicator function, the Dirichlet Kernel $(D_Nf) = \sum\limits_{-N}^{N}$, above proposition applied to $S_Nf = \sum1_{[-N,N]}(n)\hat{f}(n)e^{in\theta}$, we have \[S_Nf(\theta) = (D_N*f)(\theta)\]
\begin{lemma}
Dirichlet kernel can be equivalently expressed as \[D_N(\theta) = \frac{\sin(N+\frac{1}{2})(\theta)}{\sin(\theta/2)}\ \ \ \ \theta \neq 0, \pm 2\pi ...\]
\end{lemma}
\begin{proof}
This is proved by calculating the finite sum. Recall that $\sum_{0}^{N} ar^n = a\frac{1-r^{N+1}}{1-r}$
Let $r = e^{i\theta}$, $a = r^{-n}$
\begin{align*}
\sum\limits_{-N}^{N} r^n
&= r^{-N}\frac{1-r^{2N+1}}{1-r} \\
&= \frac{r^{-N-1/2}}{r^{-N-1/2}} r^{-N}\frac{1-r^{2N+1}}{1-r}\\
&=  \frac{r^{-N-1/2}-r^{N+1/2}}{r^{-1/2}-r^{1/2}} \\
\end{align*}
Substitute $e^{i\theta}$ back, and use $\sin(x)=\frac{e^{ix}-e^{-ix}}{2i}$ we have
\begin{align*}
\sum\limits_{-N}^{N} e^{in\theta} &= \frac{e^{(-N-1/2)i\theta}-e^{(N+1/2)i\theta}}{e^{(-1/2)i\theta}-e^{(1/2)i\theta}} \\
&= \frac{-2i\sin((N+1/2)\theta)}{-2i\sin(\theta/2)} \\
&= \frac{\sin(N+1/2)(\theta)}{\sin(\theta/2)}
\end{align*}
\end{proof}
Thus, partial sums can be rewritten as 
\[
S_Nf = \frac{1}{2\pi}\int_{\mathbb{T}}\frac{\sin(N+\frac{1}{2})(\phi)}{\sin(\phi/2)}f(\theta -\phi)d\phi
\]
A fundamental tool is given by the next theorem.
\begin{theorem}{\it Riemann Lebesgue Lemma.}
For any $f\in L^1(\mathbb{T})$, $\lim_{|n| \rightarrow \infty}\hat{f}(n) = 0$.
\end{theorem}
\begin{proof}
We first make the change of variable $\phi = \theta + \pi/n$, the interval doesn't change since for periodic functions $\int_a^{a+2\pi}f(x)dx = \int_0^{2\pi}f(x)dx$. Thus, we have
\begin{align*}
2\pi\hat{f}(n) &= \int_{\mathbb{T}}f(\theta)e^{-in\theta}d\theta \\
&= \int_{\mathbb{T}}f(\phi - \pi/n)e^{-in\phi - \pi/n}d\phi \\
&= \int_{\mathbb{T}}f(\phi - \pi/n)e^{-in\phi}e^{-i\pi}d\phi \\
&= -\int_{\mathbb{T}}f(\phi - \pi/n)e^{-in\phi}d\phi \\
\end{align*}
Adding above equation to $2\pi\hat{f}(n) = \int_{\mathbb{T}}f(\phi)e^{-in\phi}d\phi$, we have
\[
4\pi\hat{f}(n) =\int_{\mathbb{T}}(f(\phi - \pi/n)-f(\phi))e^{-in\phi}d\theta
\]
If $f$ is continuous on $\mathbb{T}$, $f$ is uniformly continuous on $\mathbb{T}$. Thus $f(\phi - \pi/n)-f(\phi) \rightarrow 0$ as $|n| \rightarrow 0$ uniformly, hence $\hat{f}(n) \rightarrow 0$. \newline
For any $f \in L^1$, we find $g$ continuous such that $|f - g|_{L^1} = \frac{1}{2\pi}\int|f(x)-g(x)|dx < \epsilon$ for any $\epsilon > 0$. By linearity, we have
\[
\hat{f}(n) = \hat{g}(n) + (\widehat{f-g})(n)
\]
We already showed $\hat{g}(n) \rightarrow 0$. Also 
\begin{align*}
|(\widehat{f-g})(n)| &= |\frac{1}{2\pi}\int_{\mathbb{T}}(f(\theta)-g(\theta))e^{-in\theta}d\theta| \\
&\leqslant \frac{1}{2\pi}\int_{\mathbb{T}}|f(\theta)-g(\theta)||e^{-in\theta}|d\theta \\
&\leqslant \epsilon
\end{align*}
Thus $\hat{f}(n) \leqslant \epsilon$ completing this proof.
\end{proof}
Now we are in the position to prove point-wise convergence theorems. First, recall that 
\[
S_Nf = \frac{1}{2\pi}\int_{\mathbb{T}}\frac{\sin(N+\frac{1}{2})(\phi)}{\sin(\phi/2)}f(\theta -\phi)d\phi
\]
We will simplify this equation in two steps: we first replace $\sin(\phi/2)$ by the function $\phi/2$, then replace the domain of integration $\mathbb{T}$ by a small interval around $\phi = 0$.
Since Dirichlet Kernel is even, we have 
\[
S_Nf = \frac{1}{2\pi}\int_{0}^{\pi}(f(\theta -\phi)+f(\theta +\phi))\frac{\sin(N+\frac{1}{2})(\phi)}{\sin(\phi/2)}d\phi
\]
Since $\frac{1}{2\pi}\int_{\mathbb{T}}\frac{\sin(N+\frac{1}{2})(\phi)}{\sin(\phi/2)}d\phi = 1$, for any constant $S$,
\[
S_Nf - S= \frac{1}{2\pi}\int_{0}^{\pi}(f(\theta -\phi)+f(\theta +\phi) - 2S)\frac{\sin(N+\frac{1}{2})(\phi)}{\sin(\phi/2)}d\phi
\]
First consider the function
\[
\phi \rightarrow \frac{1}{\sin(\phi/2)}-\frac{2}{\phi}
\]
This is bounded and continuous on $[-\pi, \pi]$. Since it is the sum of two continuous functions on $[-\pi,0)$ and $(0,\pi]$. Also, by l'Hospital's rule, the difference tends to zero at $\phi = 0$, thus it is also continuous at $\phi = 0$. Then boundedness follows. 
\begin{proposition}
The following integral goes to zero as $n \rightarrow \infty$ 
\[
\int_{0}^{\pi}(f(\theta -\phi)+f(\theta +\phi))(\frac{1}{\sin(\phi/2)}-\frac{2}{\phi})(\sin(N+\frac{1}{2}))\phi d\phi
\]
\end{proposition}
\begin{proof}
First since $f \in L^1$, $(\frac{1}{sin(2/\pi)}-\frac{2}{\phi})$ is bounded, $F_{\theta}(\phi)$ is $L^1$. Also notice that 
\[
\sin(N+\frac{1}{2})\phi = \frac{e^{i(N+\frac{1}{2})\phi} - e^{-i(N+\frac{1}{2})\phi}}{2i} = \frac{e^{iN\phi}e^{\frac{i\phi}{2}} - e^{-iN\phi}e^{\frac{i\phi}{2}}}{2i} 
\]
Riemann Lebesgue Lemma implies
\[
2\pi\hat{F}_{\theta}(\phi) = \int_{\mathbb{T}}f(\theta -\phi)+f(\theta +\phi))(\frac{1}{\sin(\phi/2)}-\frac{2}{\phi})e^{-iN\phi}d\phi \rightarrow 0
\]
Which implies
\[
\int_{\mathbb{T}}(f(\theta -\phi)+f(\theta +\phi))(\frac{1}{\sin(\phi/2)}-\frac{2}{\phi})(\sin(N+\frac{1}{2}))\phi d\phi \rightarrow 0
\]
since the integrand is an odd function, it tends to zero on $[0,\pi]$, which finishes the proof.
\end{proof}
Thus, we have reduced to proving the pointwise convergence of 
\[
\lim\limits _{N}\int_{0}^{\pi}(f(\theta -\phi)+f(\theta +\phi) - 2S)\frac{\sin(N+\frac{1}{2})(\phi)}{\phi}d\phi
\]
\begin{lemma}
\[
\lim\limits _{N}\int_{\delta}^{\pi}(f(\theta -\phi)+f(\theta +\phi) - 2S)\frac{\sin(N+\frac{1}{2})(\phi)}{\phi}d\phi = 0, \ \ \ \  \text{for small $\delta > 0$}
\]
\end{lemma}
\begin{proof}
Since $\frac{f(\theta -\phi)+f(\theta +\phi) - 2S}{\phi} \in L^1$, Riemann Lebesgue Lemma implies the desired result.
\end{proof}
\begin{theorem}
Let $f \in L^1(\mathbb{T})$, the partial sums $S_Nf(\theta)$ converges to a limit $S$ as $N \rightarrow \infty$ $\Leftrightarrow$ for some $\delta \in (0,\pi)$, we have 
\begin{align*}
\lim\limits _{N}\int_{0}^{\delta}(f(\theta -\phi)+f(\theta +\phi) - 2S)\frac{\sin(N+\frac{1}{2})(\phi)}{\phi}d\phi  = 0 \tag{*}
\end{align*}
\end{theorem}
Note: if there exists one such $\delta_0$, then it above holds for all $\delta \in (0,\pi)$ since the integral on the interval from $\delta_0,\delta$ tends to zero by Riemann Lebesgue lemma.
\begin{theorem} \textit{Dini's theorem.}
Suppose that $f$ satisfies a Dini condition at $\theta$, meaning that for some $\delta > 0$ and some real number $S$
\[
\int_0^{\delta}\frac{|f(\theta+\phi)+f(\theta-\phi)-2S|}{\phi}d\phi < \infty
\]
\end{theorem}
\begin{proof}
If Dini's condition is satisfied, we apply Riemann Lebesgue lemma to $\frac{|f(\theta+\phi)+f(\theta-\phi)-2S|}{\phi}$ to get  \[
\lim\limits _{N}\int_{0}^{\delta}(f(\theta -\phi)+f(\theta +\phi) - 2S)\frac{\sin(N+\frac{1}{2})(\phi)}{\phi}d\phi  = 0
\]
which by previous theorem implies the pointwise convergence of $S_Nf$.
\end{proof}
\begin{corollary}
Suppose that f satisfies a symmetric Hölder condition at $\theta$, meaning
\[
|f(\theta -\phi)+f(\theta +\phi)-2f(\theta)| \leqslant C|\phi|^{\alpha}
\]
for $0<\phi<\delta$, where $0<\alpha<1$. Then $\lim_NS_Nf(\theta)=f(\theta)$.
\end{corollary}
\begin{proof}
\begin{align*}
\int_0^{\delta}\frac{|f(\theta -\phi)+f(\theta +\phi)-2f(\theta)|}{\phi}d\phi &\leqslant \int_0^{\delta}C\phi^{\alpha - 1}d\phi\\
& =\frac{C}{\alpha}\phi^\alpha\mid_{0}^{\delta}\\
& =\frac{C}{\alpha}\delta^\alpha\\
& \leqslant \frac{C\pi}{\alpha}\\
& < \infty
\end{align*}
This shows that Hölder condition implies Dini condition.
\end{proof}
Before stating the main theorem, we need to introduce some tools.\\
\subsection*{Sine Integral}
Sine integral is defined to be
\[
Si(x) = \frac{2}{\pi}\int_{0}^{x}\frac{\sin t}{t}dt \ \ \ \ \text{$0 \leqslant x < \infty$}
\]
Sine integral has three properties:
\begin{itemize}
\item $Si(0) = 0$, $\lim_{x\rightarrow \infty} Si(x) = 1$.
\item $Si(x) \leqslant Si(\pi) = 1.18...$ for all $x \geqslant 0$.
\item $Si(x)$ has local maxima at $\pi$, $3\pi$, ... and local minima at $2\pi$, $4\pi$, ...
\end{itemize}  
\begin{proof}
First note that $\lim_{n\rightarrow\infty} Si(x)$ exists since 
\[\frac{\pi}{2}\lim_{n\rightarrow\infty} Si(x) = \sum_0^\infty \int_{n\pi}^{(n+1)\pi}\frac{\sin t}{t}\ \ \ \ \text{converges}\]
To find its value, by Riemann Lebesgue Lemma
\[
\lim_{n\rightarrow\infty}\int_{\mathbb{T}} (\frac{1}{\sin(\phi/2)}-\frac{2}{\phi})\sin(n+\frac{1}{2})\phi d\phi = 0
\]
And since 
\[
\lim_{n\rightarrow\infty}\int_{\mathbb{T}} \frac{\sin(n+\frac{1}{2})\phi}{\sin(\phi/2)}d\phi = \lim_{n\rightarrow\infty}\int_{\mathbb{T}} D_n(\phi)d\phi = 2\pi
\]
Also, since the integrand is even, we have 
\[
\frac{2}{\pi}\lim_{n\rightarrow\infty}\int_{0}^{\pi}\frac{\sin(n+\frac{1}{2})\phi}{\phi} d\phi = 1
\]
The left hand side is equivalent to $Si(n\pi + \frac{\pi}{2})$ by a change of variable $t = x\sin(n+1/2)$, and thus $\lim_{n \rightarrow \infty}Si(n\pi + \frac{\pi}{2}) = 1$.\\
The second and third properties is simply proved by finding where the derivative equals zero and compare local maxima values.
\end{proof}
\subsection*{Facts on Measure Theory}
We consider half intervals and one side continuous functions in the following discussion. Let h-intervals  denote the half intervals. Here we choose to use $(a,b]$ and right continuous functions. Note that the whole theory is the same if we choose to use $[a,b)$ and left continuous functions.
\begin{definition}
We define {\textbf{Lebesgue-Stieltjes measure}} associated to $F$, usually denote this by $\mu_F$, to be \[
\mu(E) = \inf\{\sum\limits_1^n[F(b_j)-F(a_j)]: E \subset \bigcup_1^{\infty}(a_j,b_j]\}
\]
An important measure,  {\textbf{Lebesgue measure}} is the complete measure $\mu_F$ associated to the function $F(x) = x$.

\end{definition}
\begin{definition}
If $F: \mathbb{R} \rightarrow \mathbb{C}$ and $x \in \mathbb{R}$. The {\textbf {total variation function}} of f is defined to be
\[
T_F(x) = \sup \sum\limits_{i=1}^N |F(x_i)-F(x_{i-1})| \ \ \ \ -\infty = x_0 < x_1<...<x_n=x\ \ n\in \mathbb{N}
\]
If $T_F(\infty) = \lim_{N\rightarrow \infty}T_F(x)$ is finite, we say $f$ is of {\textbf{bounded variation}} (on $\mathbb{R}$). We denote the space of all such $F$ by $BV$. $\sup \sum_{i=1}^N |F(x_i)-F(x_{i-1})| \ \ \ \ a = x_0 < x_1<...<x_n=b\ \ n\in \mathbb{N}$ is called the {\textbf {total variation}} of $F$ on $[a,b]$.
\end{definition}
\begin{theorem}
If $F: \mathbb{R} \rightarrow \mathbb{R}$ $F\in BV$ $\Leftrightarrow$ $F$ is the difference of two bounded increasing functions; in particular if $F\in BV$, $F = \frac{1}{2}(T_F+F) - \frac{1}{2}(T_F-F) $
\end{theorem}
Since the discontinuities of increasing functions is at most countable, thus we have the following proposition.
\begin{proposition}
If $F\in BV$, then the discontinuity of $F$ is at most countable.
\end{proposition}
For $F$ of bounded variations, $F = f^+ + f^-$, we assign a signed measure to $F$ by $\mu_F = \mu_{f^+} + \mu_{f^-}$.
We denote the integral of a function $g$ with respect to the measure $\mu_F$ by $\int gdF$ or $\int g(x)dF(x)$; such integrals are called \textbf{Lebesgue-Stieltjes} integrals.
\begin{theorem} \textit{Integral by part formula}.
If $F$ and $G$ are in $NBV$ and at least on of them is continuous, then for $-\infty < a < b < infty$, 
\[
\int_{(a,b]}FdG + \int_{(a,b]}GdF = F(b)G(b) - F(a)G(a)
\]
\end{theorem} 
\subsection*{Main Theorem}
\begin{theorem}
Suppose that $f$ is of bounded variation on $[\theta - \delta,\theta + \delta]$ for some $\delta > 0$. Then $\lim_NS_Nf(\theta) = \frac{1}{2}[f(\theta + 0)+ f(\theta - 0)].$ ($f(\theta + 0)\equiv \lim_{x\rightarrow \theta^+}f(\theta)$).
\end{theorem}
\begin{proof}
We write $m = N + \frac{1}{2}$, $F(\phi) = \frac{1}{2}(f(\theta+\phi)+f(\theta - \phi))$. $S = F(0 + 0)$; from equation (*) we have
\[
S_N - S = \frac{2}{\pi}\int_0^\delta(\frac{f(\theta+\phi)+f(\theta-\phi)}{2}-S)\frac{\sin m\phi}{\phi}d\phi + o
\]
Where $o$ denotes the error term when we performed reductions. Also $\frac{2}{\pi}\frac{\sin m\phi}{\phi}d\phi = dSi(m\phi)$, above equation becomes
\[
S_N - S = \int_0^\delta(F(\phi)-S)dSi(m\phi) + o
\]
On the interval $[0,\delta]$ $Si(x)$ and $f$ is of bounded variation, and $Si(x)$ is continuous. We set $F(\phi) = F(\phi - 0)$ where $F$ is discontinuous at $\phi$. This makes $F$ left continuous. We could do this because $F$ is of bounded variation, and we only redefined countable many points, so set of redefined points has measure zero in $\mu_{Si(m\phi)} = \frac{\sin m\phi}{\phi}\mu_{\phi}$. Integral by part implies:
\[
S_N - S = [F(\delta - 0)- S]Si(m\delta) - \int_0^\delta Si(m\phi)dF(\phi) + o
\]
$Si(\infty) = 1$ and $Si(x)$ bounded by $Si(\pi)$, so dominated convergence theorem implies:
\[
\lim_N \int_0^\delta Si(m\phi)dF(\phi) = \int_0^\delta dF(\theta) = F(\delta) - F(0)
\]
Since $F(\phi)$ is continuous at $\phi = 0$, $F(0) = F(0 + 0) = S$, $\lim_N o = 0$.
\[
\lim_N S_N - S = [F(\delta - 0)- S] - [F(\delta-0)-S] = 0
\]
\end{proof}
\end{document}